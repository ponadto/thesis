% Co musi byc w Intro:
% 1. Paragraf o MD + losowosc + collective variable.
% 4. Ze dwa slowa o klasycznym polu silowym i wkladach.
% 2. Ze wynik eksperymentu jest próbą statystyczną z pewnego rozkladu.
% 3.  Machine learning / Statistical learning method.
%  -> unsuppervised
%  -> suppervised
%  Ze unsuppervised==clustering==finding patterns in seemingly chaotic
% 5. Jakis ustep (ktory wychodzi z klasteringu) n.t. dynamicznych domen.

% Co (byc moze) powinno byc:
% 1. Metody optymalizacji poprzez alg. genetyczne.
% 2. Energia swobodna.

\chapter{Introduction}

The primary aim of this dissertation was the extraction of palpable observations from complex data describing (bio)molecular systems.
In Chapter II we present our original method of discovering quasi-rigid parts in proteins, which relies on experimental data, or from a molecular dynamics trajectory.
Chapter III presents a very different problem of identifying parts of a molecular system that propel (or hinder) a given structural transition.
In both these cases we applied unsuppervised machine learning techniques (also known as clustering), which is commonly used in finding patterns in seemingly chaotic data.

In this introductory chapter we give an overview of the molecular dynamics simulation scheme and the machine learning methodology.
The discussion here is broad, and was intended to motivate the application of clustering techniques described in the remainder of the dissertation. % TODO: inaczej
We go into more details concerning particular molecular dynamics and machine learning algorithms in Chapters II and III.


\section{Molecular dynamics}

Molecular dynamics (MD) simulations are a wide range of numerical methods designed for the study of molecular systems.
Typically, the MD approach assumes a classical potential energy function, which treats atoms as electrostatically charged points, and chemical bonds as springs and hinges.
More accurate algorithms take into account quantum effects, making the simulation more reliable, but also immensely expensive in terms of computational costs.
Although modelling intricate properties of molecular systems may seem an over-simplification, MD schemes have achieved considerable success over the recent years, and are constantly improving.

Classical MD simulations are so popular mainly because of their significantly lower computational cost.
Primarily, this allows for studying large, biologically relevant systems, such as enzymes, DNA, membrane channels, and many other biomolecules.
Secondly, the lower computational cost enables calculations of the free energy profile of the transitions associated with the function of a molecule.


The second point is especially important, because all algorithms designed for sampling the configurational space in the classical MD set-up are naturally transferable to quantum-based methods.

Underlying the simplistic, classical description of MD simulations are complex sampling techniques used for estimating macroscopic properties of microscopic systems.

TEST: \cite{gorecki2009redmd}

\subsection{Helmholtz free energy}

\subsection{Collective variable}

\section{Unsuppervised machine learning}

\subsection{Dynamic domains as clusters}

\subsection{Molecular cogs as clusters}
